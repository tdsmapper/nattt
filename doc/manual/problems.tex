\section{Problems faced while running NAT3 Daemon}

\subsection{Unknown Ethertype FE0A}
The problem is that NAT3 Daemon expects to open a TAP device, but opened a TUN device instead. Add the following flag to the makefile CFLAGS:\\
\textit{-DNAT3\_TAP}.

\subsection{NXDOMAIN not returned by BIND9 on ``A'' Record query}
In several documents associated with the daemon, it is mentioned that the application expects NXDOMAINs (Non-Existant Domains) to be returned by BIND9 when the user application attempts to do a DNS query, at which point the NAT3 daemon takes over and re-issues the query with the custom DNS Record type \footnote{See section \ref{sec:dns_rr_gen}}.

This is not required, as the NAT3 daemon checks for either a NXDOMAIN or for the absence of an answer section in the reply to the application's query. 

\subsection{BIND9 complains about unknown RR type with the custom RR type}
This was an issue that came up with a version of BIND9 unexpectedly. The solution followed at that time was to upgrade the version of BIND9 to the latest version.
